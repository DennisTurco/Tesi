\chapter{Le tecnologie utilizzate}\label{chapter:formattazione}
%
Il progetto di creazione della piattaforma di e-learning per la gestione del processo di 
OnBoarding aziendale è stato realizzato utilizzando molteplici tecnologie, servizi e linguaggi:
%
\section{Tecnologie e Servizi}\label{sec:cap_sec_subsec}
\begin{enumerate}
    \item ASP.NET MVC 6.0;
    \item Autenticazione;
    \item GitHub;
    \item Git;
\end{enumerate}
\subsection{ASP.NET MVC 6.0}\label{sec:cap_sec_subsec}
ASP.NET Core MVC è un ricco framework creato da Microsoft per la creazione di applicazioni web e API 
utilizzando il modello di progettazione Model-View-Controller.
\begin{itemize}
    \item Model: Il backend che contiene tutta la logica dei dati;
    \item View: interfaccia utente frontend o grafica (GUI);
    \item Controller: Il ``cervello'' dell'applicazione che controlla come vengono visualizzati i dati.
\end{itemize}
\subsection{Autenticazione con ``Identity''}\label{sec:cap_sec_subsec}
Siccome il progetto in questione riguarda la creazione di una piattaforma web, è stato necessario
implementare il servizio di Autenticazione attraverso il framework ASP.NET Core Identity, al
fine di otterenere non soltanto la possibilità di permettere le azioni di login e registrazione da parte degli utenti, ma
anche di farlo ottenendo una sicurezza dei dati sensibili. 
\\ \\
In particolare la forma di Autenticazione utilizzata nella piattaforma è una ``Forms Authentication'', ovvero, è quel tipo di
autenticazione nella quale l'utente deve fornire le proprie credenziali attraverso un form dedicato. 
\\ \\
ASP.NET Core Identity è un API che supporta funzionalità di login lato UI.\ Gestisce utenti,
password, dati del profilo, ruoli, email di conferma e molto altro.
In particolare, gli utenti possono creare un account con le informazioni di accesso memorizzate in Identity 
o possono utilizzare un provider di accesso esterno (es. Facebook, Google, ecc\dots ).
\\ \\
\textit{Nota:} L'algoritmo utilizzato dal servizio Identity per criptare le password è PBKDF2 (Password-Based Key Derivation Function 2):
\begin{lstlisting}[style=cs_style, caption=algoritmo di hashing per criptare le password usato dal servizio Identity]
public static string HashPassword(string password)
{
    byte[] salt;
    byte[] bytes;
    if (password == null)
    {
        throw new ArgumentNullException("password");
    }
    using (Rfc2898DeriveBytes rfc2898DeriveByte = new Rfc2898DeriveBytes(password, 16, 1000))
    {
        salt = rfc2898DeriveByte.Salt;
        bytes = rfc2898DeriveByte.GetBytes(32);
    }
    byte[] numArray = new byte[49];
    Buffer.BlockCopy(salt, 0, numArray, 1, 16);
    Buffer.BlockCopy(bytes, 0, numArray, 17, 32);
    return Convert.ToBase64String(numArray);
}
    \end{lstlisting}


\subsection{GitHub}\label{sec:cap_sec_subsec}
Il progetto durante il suo sviluppo è stato salvato in un'apposità repository privata
su GitHub.
GitHub è stato fondamentale per tenere una traccia storica di tutte le
modifiche apportate nel corso del tempo, in risposta all'esecuzione dei vari task assegnati. \\
Esso non si è limitato ad essere un semplice strumento a d'uso esclusivamente
individuale, ma ha permesso la condivisione del lavoro sia con il tutor 
aziendale che con il resto del personale.
\subsection{Git}\label{sec:cap_sec_subsec}
In parallelo all'utilizzo di GitHub è stato impiegato il software di controllo di versione Git 
con l'obbiettivo di agevolare l'aggiornamento dei file del progetto dalla workspace locale verso 
la repository privata che ospita il progetto, tramite appositi commit.
\section{Linguaggi}\label{sec:cap_sec_subsec}
\begin{enumerate}
    \item SQL;
    \item C\#;
    \item HTML, CSS, JavaScript;
\end{enumerate}
%
\subsection{SQL}\label{sec:cap_sec_subsec}
Il linguaggio SQL è stato utilizzato per 
la creazione del database del sito web. Il database contiene tutte le informazioni 
necessarie per la corretta gestione della piattaforma, tra cui i dati relativi agli utenti 
registrati, ai corsi e alle categorie dei corsi.
%
\subsection{C\#}\label{sec:cap_sec_subsec}
Il linguaggio C\# è stato utilizzato per la gestione della parte backend del sito, 
creando un dialogo tra le informazioni contenute nel database e l'interfaccia utente. 
L'accesso ai dati del database tramite il linguaggio C\# è stato possibile grazie 
al framework ``Dapper'', che consente di eseguire interrogazioni in modo efficace ed 
efficiente. In questo modo, il sito è più responsivo anche in caso di grandi quantità di 
dati da gestire, eliminando elementi come le procedure, che risultano meno 
manutenibili e più complicate da gestire a causa di una complicazione del codice nella 
parte backend per il loro utilizzo. 
\\ \\
Il linguaggio C\# è stato anche utilizzato per la 
gestione degli eventi, la navigazione tra le pagine e i controlli di vario genere.
L'utilizzo di molteplici 
tecnologie ha permesso di ottenere un prodotto completo e funzionale, in grado di 
soddisfare le esigenze dell'azienda e dei nuovi dipendenti.
%
\subsection{HTML, CSS, JavaScript}\label{sec:cap_sec_subsec}
Per la gestione dell'interfaccia utente, sono stati utilizzati il linguaggio HTML, il 
linguaggio CSS e il linguaggio JavaScript. Il linguaggio HTML è stato utilizzato per la 
creazione della struttura del sito, il linguaggio CSS per la gestione dello stile e il 
linguaggio JavaScript per la gestione degli eventi.