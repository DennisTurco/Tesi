\chapter{Le tecnologie utilizzate}\label{chapter:formattazione}
%
Il progetto di creazione della piattaforma di e-learning per la gestione del processo di 
OnBoarding aziendale è stato realizzato come web application .NET MVC 6.0 associato al servizio di Autenticazione,
utilizzando molteplici tecnologie:
%
\begin{enumerate}
    \item linguaggio SQL;
    \item linguaggio C\#;
    \item linguaggio HTML, CSS, JavaScript;
\end{enumerate}
%
\section{SQL}\label{sec:cap_sec_subsec}
In particolare, il linguaggio SQL è stato utilizzato per 
la creazione del database del sito web. Il database contiene tutte le informazioni 
necessarie per la corretta gestione della piattaforma, tra cui i dati relativi agli utenti 
registrati, ai corsi e alle categorie dei corsi.
%
\section{C\#}\label{sec:cap_sec_subsec}
Il linguaggio C\# è stato utilizzato per la gestione della parte backend del sito, 
creando un dialogo tra le informazioni contenute nel database e l'interfaccia utente. 
L'accesso ai dati del database tramite il linguaggio C\# è stato possibile grazie 
all'applicativo ``Dapper'', che consente di eseguire interrogazioni in modo efficace ed 
efficiente. In questo modo, il sito è più responsivo anche in caso di grandi quantità di 
dati da gestire, eliminando elementi come le procedure, che risultano meno 
manutenibili e più complicate da gestire a causa di una complicazione del codice nella 
parte backend per il loro utilizzo. 
\\ \\
Il linguaggio C\# è stato anche utilizzato per la 
gestione degli eventi, la navigazione tra le pagine e i controlli di vario genere.
Il progetto di creazione della piattaforma di e-learning per la gestione del processo di 
OnBoarding aziendale ha richiesto un'attenta pianificazione e la scelta di tecnologie 
adeguate per la realizzazione di un prodotto di alta qualità. L'utilizzo di molteplici 
tecnologie ha permesso di ottenere un prodotto completo e funzionale, in grado di 
soddisfare le esigenze dell'azienda e dei nuovi dipendenti.
%
\section{HTML, CSS, JavaScript}\label{sec:cap_sec_subsec}
Per la gestione dell'interfaccia utente, sono stati utilizzati il linguaggio HTML, il 
linguaggio CSS e il linguaggio JavaScript. Il linguaggio HTML è stato utilizzato per la 
creazione della struttura del sito, il linguaggio CSS per la gestione dello stile e il 
linguaggio JavaScript per la gestione degli eventi.