\chapter*{Introduzione} %Se si cambia il Titolo cambiare anche la riga successiva così che appia corretto nell'indice
\addcontentsline{toc}{chapter}{Introduzione} %Per far apparire Introduzione nell'indice (Il nome deve rispecchiare quello del chapter)
\pagenumbering{arabic} % Settaggio numerazione normale
Nel seguente elaborato tratterò una relazione relativa al progetto di tirocinio svolto 
presso un'azienda software house “ISolutions” di Noceto.
\\ \\
Il progetto si concentra sulla creazione di una piattaforma di e-learning per la 
gestione del processo di OnBoarding aziendale.\ L'obiettivo è di fornire ai nuovi 
dipendenti un'esperienza guidata, strutturata e personalizzata attraverso la 
piattaforma web. 
\\ \\
Attualmente, l'azienda gestisce tra 20 e 30 processi di OnBoarding dei nuovi 
dipendenti ogni anno, grazie al personale HR, utilizzando un foglio di calcolo Excel 
per la gestione dei dati e del completamento dei vari task.\ Tuttavia questo metodo si 
basa su un funzionamento manuale di aggiornamento dei task completati dai vari 
dipendenti e richiede una supervisione umana costante, il che lo rende inefficiente e 
non scalabile.\ Inoltre, le informazioni raccolte attraverso il modulo post-OnBoarding 
non sono sempre esaustive e dettagliate.\ A causa dell'espansione dell'azienda, 
questo sistema sta diventando sempre più inefficiente e rischioso.\ Per questo motivo 
risulta importante l'implementazione di un software dedicato al processo di 
OnBoarding dei dipendenti, con lo scopo di garantire una riduzione del rischio di 
errori e di migliorare l'esperienza generale.\ Inoltre, il software deve permettere di 
automatizzare alcune attività ripetitive, liberando tempo prezioso per il team HR, che 
potrà concentrarsi su attività di maggiore valore aggiunto per l'azienda.\ Infine, il 
nuovo sistema di OnBoarding ha la necessità di consentire l'acquisizione e 
l'archiviazione in maniera più sicura e organizzata dei dati dei dipendenti, 
rispettando le normative sulla privacy e semplificando le attività di audit interno.
\\ \\
Il processo di OnBoarding avrà una durata di circa un mese per uno sviluppatore 
junior e due settimane per uno sviluppatore senior.\ La piattaforma guiderà i nuovi 
dipendenti attraverso i vari task previsti in modo strutturato e personalizzato, 
fornendo loro gli strumenti necessari per acquisire le conoscenze e le competenze 
richieste per il loro ruolo. \\
Inoltre, la piattaforma fornirà un modulo per la raccolta di feedback post-OnBoarding. 
Il modulo sarà strutturato in modo da raccogliere informazioni dettagliate e utili per 
migliorare continuamente il processo di OnBoarding. Questo feedback sarà utilizzato 
per aggiornare e migliorare costantemente la piattaforma. \\ 
Infine, poiché l'azienda è certificata ISO 9001 (certificazione della qualità), la 
piattaforma fornirà un'ulteriore valutazione del processo di OnBoarding. Dopo che il 
dipendente ha completato il processo, verrà chiesto di esprimere un giudizio 
attraverso un modulo dedicato. Ciò permetterà all'azienda di comprendere se il 
processo di OnBoarding sta funzionando correttamente e se ci sono aree che 
possono essere migliorate. \\ 
In sintesi, la piattaforma di e-learning per la gestione del processo di OnBoarding 
aziendale rappresenta un'importante innovazione per l'azienda.\ Grazie alla sua 
efficienza e scalabilità, la piattaforma migliorerà l'esperienza del dipendente e ridurrà 
il carico di lavoro del personale HR.\ Inoltre, la raccolta di feedback dettagliati post-
OnBoarding e la valutazione attraverso il modulo dedicato forniranno all'azienda le 
informazioni necessarie per migliorare costantemente il processo di OnBoarding.
\\ \\
Il processo pricincipale di OnBoarding aziendale deve prevedere le seguenti fasi.
\begin{itemize}
    \item Introduzione (\textasciitilde{10} tasks);
    \item Setup della work station (\textasciitilde{20} tasks);
    \item Documentazione, sia amministrativa che tecnica (\textasciitilde{30} tasks);
    \item Video introduttivi, tecnici e orientati ai prodotti sviluppati (\textasciitilde{10} tasks);
    \item Overview dell'organizzazione aziendale e degli strumenti utilizzati (\textasciitilde{10} tasks);
    \item Hands On degli applicativi aziendali (\textasciitilde{20} tasks);
    \item Formazione sui principali linguaggi di programmazione utilizzati (\textasciitilde{20} tasks);
    \item Overview struttura codice delle soluzioni software (\textasciitilde{10} tasks);
    \item Test finale con revisione e valutazione.
\end{itemize}
Sono previsti inoltre alcuni step preliminari, in carico al team HR, per la predispozione di quanto 
necessario (creazione utenza, preparazione pc etc\dots), e una fase finale di 
retrospettiva per raccogliere feedback riguardo il processo dal neo assunto.\ 
\\ \\
Alcune desiderate del progetto:
\begin{itemize}
    \item Integrazione autenticazione con modulo di login e gestione livelli di autorizzazione;
    \item Pannello amministrativo per il team HR per gestire (creazione, modifica, eliminazione) dei vari tasks;
    \item Tracciamento tempo impiegato sui vari task e reportistica per team HR;\
    \item Possibilità di avviare, sospendere e saltare un task;
    \item Integrazione fase finale di test, revisione e valutazione.
\end{itemize}
Il progetto di creazione della piattaforma di e-learning per la gestione del processo di
OnBoarding aziendale è stato realizzato come web application .NET MVC 6.0
utilizzando molteplici tecnologie e linguaggi di programmazione tra cui: C\#, HTML, 
Css, Javascript, SQL.\
\\
\begin{itemize}
    \item Capitolo 1 → Le caratteristiche dell'OnBoarding. 
    \item Capitolo 2 → Il modello dell'OnBoarding nell'azienda.
    \item Capitolo 3 → Le tecnologie utilizzate.
    \item Capitolo 4 → Il progetto ed il suo sviluppo.
    \item Capitolo 5 → Risultati e sviluppi futuri.
\end{itemize}
