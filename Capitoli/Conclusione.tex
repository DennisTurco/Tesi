\chapter*{Conclusione} %Se si cambia il Titolo cambiare anche la riga successiva così che appia corretto nell'conclusione
\addcontentsline{toc}{chapter}{Conclusione} %Per far apparire Introduzione nell'indice (Il nome deve rispecchiare quello del chapter)
%
In conclusione, durante il mio tirocinio, ho sviluppato un'applicazione web basata su .NET MVC 6.0 con 
l'obiettivo di creare un servizio che automatizzasse in modo efficiente il complesso 
processo di integrazione dei nuovi dipendenti, un passaggio cruciale e ricorrente per 
tutti i nuovi membri dell'organico aziendale.\ Questo progetto ambizioso ha consentito 
all'azienda di mettere a disposizione uno strumento estremamente automatizzato, riducendo 
significativamente il carico di lavoro del dipartimento HR e delegando gran parte delle 
attività operative alla gestione automatizzata del sistema.
\\ \\
Va notato che l'azienda, in virtù della sua certificazione ISO 9001, non solo è obbligata a farlo, 
ma si impegna attivamente a gestire e analizzare attentamente i feedback provenienti dagli utenti del sistema, 
con l'obiettivo di migliorarne costantemente l'efficacia e l'efficienza.
\\ \\
L'applicazione web, pur rispettando appieno tutti i requisiti aziendali iniziali, 
offre un ampio margine per ulteriori miglioramenti, come discusso dettagliatamente nel capitolo 5 del documento~\ref{chapter:Risultati_e_sviluppi_futuri}, 
offrendo così opportunità significative per un futuro sviluppo e ottimizzazione.