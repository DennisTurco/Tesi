\documentclass{theme/uniprthesis}

%%%%%%%%%%%Some Extra Packages%%%%%%%%%%%
\usepackage[italian]{babel}		% To have Italina names in Sections, Figures, Chapters etc.
\usepackage{todonotes}			% To ease the revision

\usepackage{blindtext} 			% Dummy Text - remove

\usepackage{hyperref}			% link library
\usepackage{listings}
\usepackage{color}
\usepackage{xcolor}
\usepackage{float}% http://ctan.org/pkg/float
%%%%%%%%%%%%%%%%%%%%%%%%%%%%%%%%%


%%%%%%%%%%%%Custom%%%%%%%%%%%%%%%%%%%%%
% inline code
\definecolor{codebackground}{HTML}{ededeb}
\newcommand{\inlinecode}[1]{\colorbox{codebackground}{\textcolor{red}{#1}}}

\hypersetup{
    colorlinks=true,
    linkcolor=blue,
    filecolor=magenta,      
    urlcolor=cyan,
    pdfpagemode=FullScreen,
    }

\urlstyle{same}

%TODO: fixhere
% SQL Sintax highlighting
\definecolor{dkgreen}{rgb}{0,0.6,0}
\definecolor{gray}{rgb}{0.5,0.5,0.5}
\definecolor{mauve}{rgb}{0.58,0,0.82}
\definecolor{backcolour}{rgb}{1, 0.98, 0.98}
\lstset{language=SQL,
    backgroundcolor=\color{backcolour},
    basicstyle={\small\ttfamily},
    breakatwhitespace=false,
    breaklines=true,
    captionpos=b,
    classoffset=0,
    columns=flexible,
    commentstyle=\color{dkgreen},
    frameshape={}{}{}{}, %To remove to vertical lines on left, set `frameshape={}{}{}{}`
    keepspaces=true,
    keywordstyle=\color{blue},
    numbers=left,                    
    numbersep=5pt,    
    numberstyle=\tiny\color{gray},
    showstringspaces=false,
    showspaces=false,
    showtabs=false,
    stringstyle=\color{mauve},
    tabsize=2,
}

%TODO: fixhere
% Code Sintax highlighting
\definecolor{bluekeywords}{rgb}{0,0,1}
\definecolor{greencomments}{rgb}{0,0.5,0}
\definecolor{redstrings}{rgb}{0.64,0.08,0.08}
\definecolor{xmlcomments}{rgb}{0.5,0.5,0.5}
\definecolor{types}{rgb}{0.17,0.57,0.68}
\lstset{language=[Sharp]C,
    commentstyle=\color{greencomments},
    morekeywords={partial, var, value, get, set},
    keywordstyle=\color{bluekeywords},
    stringstyle=\color{redstrings},
    basicstyle=\ttfamily\small,
}



%%%%%%%%%%%%%%%%%%%%%%%%%%%%%%%%%


%%%%% THESIS / TITLE PAGE INFORMATION
% Everybody needs to complete the following:

\title{Titolo in Italiano}
\author{Dennis Turco}
\advisor{Prof.\ Roberto Alfieri}
\college{Dipartimento di Scienze Matematiche, Fisiche e Informatiche}
\degree{Corso di Laurea [Triennale in Informatica]}
\degreeyears{2020--2023}


% Not mandatory fields
\newcommand{\subTitle}{Titolo in Inglese} %Subtitle, usually the english version of the title

%\newcommand{\advisorSecond}{Prof. Nome2 Cognome2} % For multiple (up to 4) advisors -- if this is not present then also the remaining ones are automatically omitted
%\newcommand{\advisorThird}{Dott. Nome3 Cognome3} % For multiple (up to 4) advisors -- if this is not present then also the remaining ones are automatically omitted
%\newcommand{\advisorFourth}{Dott. Nome4 Cognome4} % For multiple (up to 4) advisors

\newcommand{\coadvisor}{Prof.\ co-Nome co-Cognome} %For multiple (up to 4) coadvisors -- if this is not present then also the remaining ones are automatically omitted
\newcommand{\coadvisorSecond}{Prof.\ co-Nome2 co-Cognome2} % For multiple (up to 4) coadvisors -- if this is not present then also the remaining ones are automatically omitted
%\newcommand{\coadvisorThird}{Dott. co-Nome3 co-Cognome3} % For multiple (up to 4) coadvisors -- if this is not present then also the remaining ones are automatically omitted
%\newcommand{\coadvisorFourth}{Dott. co-Nome4 co-Cognome4} % For multiple (up to 4) coadvisors


\begin{document}

\maketitle

%%%% La dedica
% \newpage
% \thispagestyle{empty}
% \null\vspace{\stretch{1}}
% \begin{flushright}
% 	\textit{Dedica}
% \end{flushright}
% \vspace{\stretch{3}}\null
% \newpage

%%%% Gli indici
\pagestyle{plain}
\pagenumbering{roman}
\tableofcontents
%
\listoffigures    %Commentare se non vi sono Immagini
\listofalgorithms %Commentare se non vi sono Algoritmi
\listoftables     %Commentare se non vi sono Tabelle
%
%
%
%%%% La prefazione
\chapter*{Introduzione}\label{chapter:introduzione} %Se si cambia il Titolo cambiare anche la riga successiva così che appia corretto nell'indice
\addcontentsline{toc}{chapter}{Introduzione} %Per far apparire Introduzione nell'indice (Il nome deve rispecchiare quello del chapter)
\pagenumbering{arabic} % Settaggio numerazione normale
Nel seguente elaborato tratterò una relazione relativa al progetto di tirocinio svolto 
presso un'azienda software house “iSolutions” di Noceto.
\\ \\
Il progetto si concentra sulla creazione di una piattaforma di e-learning per la 
gestione del processo di OnBoarding aziendale.\ L'obiettivo è di fornire ai nuovi 
dipendenti un'esperienza guidata, strutturata e personalizzata attraverso la 
piattaforma web. 
\\ \\
Attualmente, l'azienda gestisce tra 20 e 30 processi di OnBoarding dei nuovi 
dipendenti ogni anno, grazie al personale HR, utilizzando un foglio di calcolo Excel 
per la gestione dei dati e del completamento dei vari task.\ Tuttavia questo metodo si 
basa su un funzionamento manuale di aggiornamento dei task completati dai vari 
dipendenti e richiede una supervisione umana costante, il che lo rende inefficiente e 
non scalabile.\ Inoltre, le informazioni raccolte attraverso il modulo post-OnBoarding 
non sono sempre esaustive e dettagliate.\ A causa dell'espansione dell'azienda, 
questo sistema sta diventando sempre più inefficiente e rischioso.\ Per questo motivo 
risulta importante l'implementazione di un software dedicato al processo di 
OnBoarding dei dipendenti, con lo scopo di garantire una riduzione del rischio di 
errori e di migliorare l'esperienza generale.\ Inoltre, il software deve permettere di 
automatizzare alcune attività ripetitive, liberando tempo prezioso per il team HR, che 
potrà concentrarsi su attività di maggiore valore aggiunto per l'azienda.\ Infine, il 
nuovo sistema di OnBoarding ha la necessità di consentire l'acquisizione e 
l'archiviazione in maniera più sicura e organizzata dei dati dei dipendenti, 
rispettando le normative sulla privacy e semplificando le attività di audit interno.
\\ \\
Il processo di OnBoarding avrà una durata di circa un mese per uno sviluppatore 
junior e due settimane per uno sviluppatore senior.\ La piattaforma guiderà i nuovi 
dipendenti attraverso i vari task previsti in modo strutturato e personalizzato, 
fornendo loro gli strumenti necessari per acquisire le conoscenze e le competenze 
richieste per il loro ruolo. \\
Inoltre, la piattaforma fornirà un modulo per la raccolta di feedback post-OnBoarding. 
Il modulo sarà strutturato in modo da raccogliere informazioni dettagliate e utili per 
migliorare continuamente il processo di OnBoarding. Questo feedback sarà utilizzato 
per aggiornare e migliorare costantemente la piattaforma. \\ 
Infine, poiché l'azienda è certificata ISO 9001 (certificazione della qualità), la 
piattaforma fornirà un'ulteriore valutazione del processo di OnBoarding. Dopo che il 
dipendente ha completato il processo, verrà chiesto di esprimere un giudizio 
attraverso un modulo dedicato. Ciò permetterà all'azienda di comprendere se il 
processo di OnBoarding sta funzionando correttamente e se ci sono aree che 
possono essere migliorate. \\ 
In sintesi, la piattaforma di e-learning per la gestione del processo di OnBoarding 
aziendale rappresenta un'importante innovazione per l'azienda.\ Grazie alla sua 
efficienza e scalabilità, la piattaforma migliorerà l'esperienza del dipendente e ridurrà 
il carico di lavoro del personale HR.\ Inoltre, la raccolta di feedback dettagliati post-OnBoarding
 e la valutazione attraverso il modulo dedicato forniranno all'azienda le 
informazioni necessarie per migliorare costantemente il processo di OnBoarding.
\\ \\
Il processo pricincipale di OnBoarding aziendale deve prevedere le seguenti fasi.
\begin{itemize}
    \item Introduzione (\textasciitilde{10} tasks);
    \item Setup della work station (\textasciitilde{20} tasks);
    \item Documentazione, sia amministrativa che tecnica (\textasciitilde{30} tasks);
    \item Video introduttivi, tecnici e orientati ai prodotti sviluppati (\textasciitilde{10} tasks);
    \item Overview dell'organizzazione aziendale e degli strumenti utilizzati (\textasciitilde{10} tasks);
    \item Hands On degli applicativi aziendali (\textasciitilde{20} tasks);
    \item Formazione sui principali linguaggi di programmazione utilizzati (\textasciitilde{20} tasks);
    \item Overview struttura codice delle soluzioni software (\textasciitilde{10} tasks);
    \item Test finale con revisione e valutazione.
\end{itemize}
Sono previsti inoltre alcuni step preliminari, in carico al team HR, per la predispozione di quanto 
necessario (creazione utenza, preparazione pc etc\dots), e una fase finale di 
retrospettiva per raccogliere feedback riguardo il processo dal neo assunto.\ 
\\ \\
Alcune desiderate del progetto:
\begin{itemize}
    \item Integrazione autenticazione con modulo di login e gestione livelli di autorizzazione;
    \item Pannello amministrativo per il team HR per gestire (creazione, modifica, eliminazione) dei vari tasks;
    \item Tracciamento tempo impiegato sui vari task e reportistica per team HR;\
    \item Possibilità di avviare, sospendere e saltare un task;
    \item Integrazione fase finale di test, revisione e valutazione.
\end{itemize}
Il progetto di creazione della piattaforma di e-learning per la gestione del processo di
OnBoarding aziendale è stato realizzato come web application .NET MVC 6.0
utilizzando molteplici tecnologie e linguaggi di programmazione tra cui: C\#, HTML, 
Css, Javascript, SQL.

%
%%%% I Capitoli di Contenuto	
\pagestyle{fancy}
\chapter{Le caratteristiche dell'OnBoarding}\label{chapter:caratteristiche_onboarding}
Il processo di OnBoarding, sebbene possa inizialmente apparire come una fase apparentemente semplice e marginale, 
riveste in realtà un'importanza cruciale nel contesto dell'integrazione dei nuovi membri nell'organico di ISolutions.\
Come precedentemente introdotto nella sezione iniziale di questo elaborato~\ref{chapter:introduzione}, %TODO fixhere
il processo di OnBoarding è tipicamente strutturato su un periodo di circa un mese, ma è fondamentale sottolineare che 
la sua durata può variare considerevolmente da un dipendente all'altro.\ Questo periodo iniziale di integrazione, 
che potrebbe superficialmente sembrare un'appendice, svolge in realtà una serie di funzioni di rilevanza vitale.
\\ \\
È importante notare che il processo di OnBoarding viene coordinato e gestito con grande cura da un team HR altamente specializzato.\ 
Questo team opera con precisione e attenzione ai dettagli per garantire una transizione efficace e senza intoppi per i nuovi arrivati.\ 
La sua importanza ricade con lo scopo di offrire una garanzia di un'ottimale integrazione dei nuovi dipendenti nell'ambiente aziendale.
\\ \\
Risulta evidente che il processo di OnBoarding richiede un approccio attento e uniforme da parte di tutto il personale aziendale.\ 
Questa necessità è strettamente correlata alla complessità delle operazioni svolte da ISolutions e alla diversità dei numerosi 
strumenti e tecnologie che l'azienda utilizza, compresi quelli di natura proprietaria.\ Questi strumenti rivestono un ruolo 
centrale sia nella gestione che nell'analisi dei dati aziendali e nei processi operativi nel loro complesso.\ 
Inoltre, essi giocano un ruolo significativo nella consegna di servizi di alta qualità ai clienti dell'azienda.
\\ \\
In questo contesto, diventa chiaro che un nuovo membro del team deve affrontare l'obiettivo di familiarizzarsi 
con tali tecnologie e procedure aziendali.\ Questo obiettivo viene raggiunto attraverso il processo di OnBoarding, 
che svolge il ruolo di guida nella presentazione di tutte le conoscenze iniziali necessarie per lavorare in modo produttivo 
all'interno del sistema aziendale. Durante questa fase, i nuovi arrivati avranno l'opportunità di acquisire familiarità con 
gli strumenti, i linguaggi, le convenzioni aziendali e i protocolli che sono essenziali per contribuire al successo e 
all'efficienza dell'azienda.
\\ \\
Pertanto, il processo di OnBoarding, lungi dall'essere una mera formalità, rappresenta un elemento fondamentale per il successo 
e il corretto funzionamento di ISolutions e deve essere affrontato con la massima serietà e impegno da parte di tutti i suoi attori.
\\ \\
Il processo di OnBoarding è un'attività complessa che si pone diversi obiettivi, 
e pertanto è strutturato in diverse sezioni con l'intento di fornire ai nuovi 
dipendenti una conoscenza iniziale completa.\ Le sezioni principali del processo di OnBoarding si suddividono in:
%TODO aggiungere le varie categorie e descriverle
\begin{itemize}
    \item Video;
    \item Documentazione;
    \item Hands On;
    \item Audit;
\end{itemize}
%
Al momento, come precedentemente accennato, il processo di OnBoarding non fa uso di un sistema automatizzato; 
al contrario, è gestito attraverso un semplice foglio Excel. Questo foglio è consultato dai dipendenti interessati 
e viene modificato manualmente quando il dipendente termina le relative attività interne da parte del personale HR incaricato.\ 
Tuttavia, questo approccio presenta alcune limitazioni.\ 
Innanzitutto, non permette di raccogliere e organizzare dati statistici derivati dall'uso del processo.\ 
In secondo luogo, non offre la possibilità agli utenti di interagire direttamente con il sistema.
\\ \\
Ciò significa che, con l'attuale modalità, manca l'opportunità di ottenere una visione dettagliata e sistematica dell'andamento del processo di OnBoarding, 
impedendo all'azienda di identificare potenziali aree di miglioramento o di misurare l'efficacia del processo stesso.\ 
Inoltre, l'assenza di un'interazione diretta con il sistema può ostacolare la comunicazione e la partecipazione attiva dei dipendenti nel corso 
del processo, rendendo l'integrazione meno fluida e coinvolgente.
\\ \\
Pertanto, esiste un evidente margine per l'implementazione di un sistema automatizzato di OnBoarding che non solo semplificherebbe 
la gestione, ma consentirebbe anche la raccolta e l'analisi di dati utili per ottimizzare il processo e coinvolgere in modo più 
attivo i dipendenti coinvolti.
%TODO inserire immagine esempio di un template del processo di onboarding
% \begin{figure}[ht]
% 	\centering
% 	\includegraphics[width=0.7\textwidth]{img/charts.png}
% 	\caption{struttura del file excel di OnBoarding}
% 	\label{fig:charts}
% \end{figure}
\chapter{Il modello dell'OnBoarding nell'azienda}\label{chapter:formattazione}
\chapter{Le tecnologie utilizzate}\label{chapter:formattazione}
%
Il progetto di creazione della piattaforma di e-learning per la gestione del processo di 
OnBoarding aziendale è stato realizzato utilizzando molteplici tecnologie, servizi e linguaggi:
%
\section{Tecnologie e Servizi}\label{sec:cap_sec_subsec}
\begin{enumerate}
    \item ASP.NET MVC 6.0;
    \item Autenticazione;
    \item GitHub;
    \item Git;
\end{enumerate}
\subsection{ASP.NET MVC 6.0}\label{sec:cap_sec_subsec}
ASP.NET Core MVC è un ricco framework creato da Microsoft per la creazione di applicazioni web e API 
utilizzando il modello di progettazione Model-View-Controller.
\begin{itemize}
    \item Model: Il backend che contiene tutta la logica dei dati;
    \item View: interfaccia utente frontend o grafica (GUI);
    \item Controller: Il ``cervello'' dell'applicazione che controlla come vengono visualizzati i dati.
\end{itemize}
\subsection{Autenticazione con ``Identity''}\label{sec:cap_sec_subsec}
Siccome il progetto in questione riguarda la creazione di una piattaforma web, è stato necessario
implementare il servizio di Autenticazione attraverso il framework ASP.NET Core Identity, al
fine di otterenere non soltanto la possibilità di permettere le azioni di login e registrazione da parte degli utenti, ma
anche di farlo ottenendo una sicurezza dei dati sensibili. 
\\ \\
In particolare la forma di Autenticazione utilizzata nella piattaforma è una ``Forms Authentication'', ovvero, è quel tipo di
autenticazione nella quale l'utente deve fornire le proprie credenziali attraverso un form dedicato. 
\\ \\
ASP.NET Core Identity è un API che supporta funzionalità di login lato UI.\ Gestisce utenti,
password, dati del profilo, ruoli, email di conferma e molto altro.
In particolare, gli utenti possono creare un account con le informazioni di accesso memorizzate in Identity 
o possono utilizzare un provider di accesso esterno (es. Facebook, Google, ecc\dots ).
\\ \\
\textit{Nota:} L'algoritmo utilizzato dal servizio Identity per criptare le password è PBKDF2 (Password-Based Key Derivation Function 2):
\begin{lstlisting}[style=cs_style, caption=algoritmo di hashing per criptare le password usato dal servizio Identity]
public static string HashPassword(string password)
{
    byte[] salt;
    byte[] bytes;
    if (password == null)
    {
        throw new ArgumentNullException("password");
    }
    using (Rfc2898DeriveBytes rfc2898DeriveByte = new Rfc2898DeriveBytes(password, 16, 1000))
    {
        salt = rfc2898DeriveByte.Salt;
        bytes = rfc2898DeriveByte.GetBytes(32);
    }
    byte[] numArray = new byte[49];
    Buffer.BlockCopy(salt, 0, numArray, 1, 16);
    Buffer.BlockCopy(bytes, 0, numArray, 17, 32);
    return Convert.ToBase64String(numArray);
}
    \end{lstlisting}


\subsection{GitHub}\label{sec:cap_sec_subsec}
Il progetto durante il suo sviluppo è stato salvato in un'apposità repository privata
su GitHub.
GitHub è stato fondamentale per tenere una traccia storica di tutte le
modifiche apportate nel corso del tempo, in risposta all'esecuzione dei vari task assegnati. \\
Esso non si è limitato ad essere un semplice strumento a d'uso esclusivamente
individuale, ma ha permesso la condivisione del lavoro sia con il tutor 
aziendale che con il resto del personale.
\subsection{Git}\label{sec:cap_sec_subsec}
In parallelo all'utilizzo di GitHub è stato impiegato il software di controllo di versione Git 
con l'obbiettivo di agevolare l'aggiornamento dei file del progetto dalla workspace locale verso 
la repository privata che ospita il progetto, tramite appositi commit.
\section{Linguaggi}\label{sec:cap_sec_subsec}
\begin{enumerate}
    \item SQL;
    \item C\#;
    \item HTML, CSS, JavaScript;
\end{enumerate}
%
\subsection{SQL}\label{sec:cap_sec_subsec}
Il linguaggio SQL è stato utilizzato per 
la creazione del database del sito web. Il database contiene tutte le informazioni 
necessarie per la corretta gestione della piattaforma, tra cui i dati relativi agli utenti 
registrati, ai corsi e alle categorie dei corsi.
%
\subsection{C\#}\label{sec:cap_sec_subsec}
Il linguaggio C\# è stato utilizzato per la gestione della parte backend del sito, 
creando un dialogo tra le informazioni contenute nel database e l'interfaccia utente. 
L'accesso ai dati del database tramite il linguaggio C\# è stato possibile grazie 
al framework ``Dapper'', che consente di eseguire interrogazioni in modo efficace ed 
efficiente. In questo modo, il sito è più responsivo anche in caso di grandi quantità di 
dati da gestire, eliminando elementi come le procedure, che risultano meno 
manutenibili e più complicate da gestire a causa di una complicazione del codice nella 
parte backend per il loro utilizzo. 
\\ \\
Il linguaggio C\# è stato anche utilizzato per la 
gestione degli eventi, la navigazione tra le pagine e i controlli di vario genere.
L'utilizzo di molteplici 
tecnologie ha permesso di ottenere un prodotto completo e funzionale, in grado di 
soddisfare le esigenze dell'azienda e dei nuovi dipendenti.
%
\subsection{HTML, CSS, JavaScript}\label{sec:cap_sec_subsec}
Per la gestione dell'interfaccia utente, sono stati utilizzati il linguaggio HTML, il 
linguaggio CSS e il linguaggio JavaScript. Il linguaggio HTML è stato utilizzato per la 
creazione della struttura del sito, il linguaggio CSS per la gestione dello stile e il 
linguaggio JavaScript per la gestione degli eventi.
\chapter{Il progetto e il suo sviluppo}\label{chapter:formattazione}
%
\section{Database}\label{sec:cap_sec_subsec}
Una parte sostanziale del nostro impegno aziendale per portare a compimento il
progetto si è concentrata in modo dettagliato sulla creazione del database,
ritenuto elemento cruciale per assicurare il pieno funzionamento della
piattaforma. \\ \\ La creazione del database è stata articolata in diverse fasi
chiave:
\begin{itemize}
	\item Progettazione della struttura del database, un processo attentamente studiato.
	\item Scrittura del database in linguaggio SQL, una tappa essenziale.
	\item Implementazione delle interrogazioni al database nella sezione di backend,
	      facendo uso della libreria Dapper.
\end{itemize}
\textit{Nota}: il database è stato creato e gestito unicamente in locale per questioni di semplicità e per
l'esecuzione di test sulla correttezza della struttura e dell'implementazione del database stesso.
%
%
\subsection{Progettazione}
La fase di progettazione~\ref{fig:one} del database ha occupato un ruolo
fondamentale nel corso di questo processo, coinvolgendo un'analisi costante e
una riflessione profonda. Il nostro obiettivo principale era plasmare un
database estremamente completo, in grado di soddisfare appieno le esigenze
della piattaforma di e-learning. In aggiunta, ci siamo concentrati su:
\begin{itemize}
	\item Migliorare la leggibilità del database;
	\item Rendere più agevole la manutenzione;
	\item Fornire ampie possibilità di estensione del sistema.
\end{itemize}
\begin{figure}[ht]
	\centering
	\includegraphics[width=1\textwidth]{img/progettazione_database.png}
	\caption{progettazione del database}
	\label{fig:one}
\end{figure}
\textit{Nota}: la progettazione mostrata non tiene traccia delle tabelle utente, perchè vengono
gestite automaticamente dal progetto .NET grazie al servizio di autenticazione già
incluso (libreria \inlinecode{Microsoft.EntityFrameworkCore.Migrations;}).
%
\subsection{Scrittura nel linguaggio SQL}
Il risultato finale di questa fase è il seguente (non sono riportate le tabelle
generate dal servizio di Autenticazione, perchè gestite automaticamente dalla
libreria a disposizione):
\begin{lstlisting}[language=SQL, caption=traduzione della progettazione del database nel linguaggio SQL]
CREATE TABLE [dbo].[Course] ( 
    [Id]         INT            IDENTITY (1, 1) NOT NULL, 
    [Name]       NVARCHAR (255) NOT NULL, 
    [Completion] REAL           DEFAULT ((0)) NULL, 
    [StartDate]  DATETIME2 (7)  NULL, 
    [EndDate]    DATETIME2 (7)  NULL, 
    CONSTRAINT [PK_Course] PRIMARY KEY CLUSTERED ([Id] ASC) 
); 

CREATE TABLE [dbo].[Category] ( 
    [Id]         INT            IDENTITY (1, 1) NOT NULL, 
    [Name]       NVARCHAR (255) NOT NULL, 
    [Completion] REAL           DEFAULT ((0)) NULL, 
    [IdCourse]   INT            NOT NULL, 
    CONSTRAINT [PK_Category] PRIMARY KEY CLUSTERED ([Id] ASC), 
    CONSTRAINT [FK_Category_Course_IdCourse]
	FOREIGN KEY ([IdCourse]) REFERENCES [dbo].[Course] ([Id]) 
	ON DELETE CASCADE 
   ); 
   GO 
   CREATE NONCLUSTERED INDEX [IX_Category_IdCourse] 
	   ON [dbo].[Category]([IdCourse] ASC); 	

-- 3 stati: 
	-- 1. done 
	-- 2. todo 
	-- 3. check 
CREATE TABLE [dbo].[StepStatus] ( 
	[Id]     INT            IDENTITY (1, 1) NOT NULL, 
	[Status] NVARCHAR (10) NOT NULL, 
	CONSTRAINT [PK_StepStatus] PRIMARY KEY CLUSTERED ([Id] ASC) 
); 
	
CREATE TABLE [dbo].[Step] ( 
	[Id]             INT            IDENTITY (1, 1) NOT NULL, 
	[Name]           NVARCHAR (255) NOT NULL, 
	[Description]    NVARCHAR (MAX) NULL, 
	[StartDate]      DATETIME2 (7)  NULL, 
	[EndDate]        DATETIME2 (7)  NULL, 
	[Lock]           BIT            DEFAULT ((1)) NULL, 
	[NeedValidation] BIT            DEFAULT ((0)) NULL, 
	[IdStatus]       INT            DEFAULT ((1)) NULL, 
	[IdCategory]     INT            NOT NULL, 
	CONSTRAINT [PK_Step] PRIMARY KEY CLUSTERED ([Id] ASC), 
	CONSTRAINT [FK_Step_StepStatus_IdStatus]  
FOREIGN KEY ([IdStatus]) REFERENCES [dbo].[StepStatus] ([Id]) 
ON DELETE CASCADE, 
	CONSTRAINT [FK_Step_Category_IdCategory]  
FOREIGN KEY ([IdCategory]) REFERENCES [dbo].[Category] ([Id]) 
ON DELETE CASCADE 
); 
GO 
CREATE NONCLUSTERED INDEX [IX_Step_IdCategory] 
	ON [dbo].[Step]([IdCategory] ASC); 
GO 
CREATE NONCLUSTERED INDEX [IX_Step_IdStatus] 
	ON [dbo].[Step]([IdStatus] ASC); 

CREATE TABLE [dbo].[UserCourse] ( 
	[UserId]   NVARCHAR (450) NOT NULL, 
	[CourseId] INT            NOT NULL, 
	CONSTRAINT [FK_UserCourse_Course_CourseModel]  
FOREIGN KEY ([CourseId]) REFERENCES [dbo].[Course] ([Id]) 
ON DELETE CASCADE, 
	CONSTRAINT [FK_UserCourse_AspNetUsers_UserId]  
FOREIGN KEY ([UserId]) REFERENCES [dbo].[AspNetUsers] ([Id]) 
ON DELETE CASCADE 
); 
GO 
CREATE NONCLUSTERED INDEX [IX_UserCourse_CourseModel] 
    ON [dbo].[UserCourse]([CourseId] ASC); 
GO 
CREATE NONCLUSTERED INDEX [IX_UserCourse_UserId] 
    ON [dbo].[UserCourse]([UserId] ASC); 
 
CREATE TABLE [dbo].[CourseTemplate] ( 
    [Id]             INT            IDENTITY (1, 1) NOT NULL, 
    [Name]           NVARCHAR (255) NOT NULL, 
    [Creator]        NVARCHAR (255) NOT NULL, 
    [CreationDate]   DATETIME2 (7)  NULL, 
    [LastUpdateDate] DATETIME2 (7)  NULL, 
    CONSTRAINT [PK_CourseTemplate] PRIMARY KEY CLUSTERED ([Id] ASC) 
); 
 
CREATE TABLE [dbo].[CategoryTemplate] ( 
    [Id]                INT         IDENTITY (1, 1) NOT NULL, 
    [name]              NCHAR (255) NOT NULL, 
    IDCourseTemplate INT         NOT NULL, 
    PRIMARY KEY CLUSTERED ([Id] ASC), 
    FOREIGN KEY (IDCourseTemplate)  
 REFERENCES [dbo].[CourseTemplate] ([Id]) 
 ON DELETE CASCADE 
); 
 
CREATE TABLE [dbo].[StepTemplate] ( 
    [Id]                 INT            IDENTITY (1, 1) NOT NULL, 
    [Name]               NVARCHAR (255) NOT NULL, 
    [Description]        NVARCHAR (MAX) NOT NULL, 
    [NeedValidation]     BIT            DEFAULT ((0)) NULL, 
    [IDCategoryTemplate] INT            NOT NULL, 
    CONSTRAINT [PK_StepTemplate] PRIMARY KEY CLUSTERED ([Id] ASC), 
    CONSTRAINT [FK_StepTemplate_CategoryTemplate_IDCategoryTemplate]  
 FOREIGN KEY ([IDCategoryTemplate]) REFERENCES [dbo].[CategoryTemplate] ([Id]) 
 ON DELETE CASCADE 
); 
GO 
CREATE NONCLUSTERED INDEX [IX_StepTemplate_IDCategoryTemplate] 
    ON [dbo].[StepTemplate]([IDCategoryTemplate] ASC);
\end{lstlisting}
%
\subsection{Interrogazioni al Database}
Inizialmente, le interrogazioni al database erano state sviluppate attraverso
\href{https://learn.microsoft.com/it-it/sql/relational-databases/stored-procedures/stored-procedures-database-engine?view=sql-server-ver16}{stored
	procedure}.\ Tra i numerosi vantaggi proposti, l'idea principale era ottenere:
\begin{itemize}
	\item Riutilizzo del codice;
	\item Semplificazione della manutenzione;
	\item Prestazioni migliorate;
\end{itemize}
Tuttavia, durante lo sviluppo dell'applicativo, grazie al suggerimento di alcuni membri
dell'azienda, si è preferito sostituire le procedure con interrogazioni dirette dal codice
tramite il framework \href{https://learn.microsoft.com/it-it/azure/azure-sql/database/elastic-scale-working-with-dapper?view=azuresql}{Dapper}, al fine di poter migliorare:
\begin{itemize}
	\item Prestazioni in termini di tempo;
	\item Semplificare il debugging del codice;
	\item Centrallizzare l'intera logica in un unico posto.
\end{itemize}
All'interno della piattaforma si possono distinguere 3 macro categorie di interrogazioni per:
\begin{itemize}
	\item Inserimento dei dati per aggiungere:
	      \begin{itemize}
		      \item corsi;
		      \item categorie (sotto tipo dei corsi);
		      \item step (sotto tipo delle categorie);
		      \item corsi template;
		      \item categorie template (sotto tipo dei corsi template);
		      \item step template (sotto tipo delle categorie template).
	      \end{itemize}
	\item Eliminazione dei dati per togliere:
	      \begin{itemize}
		      \item corsi;
		      \item categorie (sotto tipo dei corsi);
		      \item step (sotto tipo delle categorie);
		      \item corsi template;
		      \item categorie template (sotto tipo dei corsi template);
		      \item step template (sotto tipo delle categorie template).
	      \end{itemize}
	\item Lettura dei dati per permettere di ottenere tutte le informazioni necessarie
	      per la generazione corretta della pagina dinamica dato un determinato utente
	      connesso;
\end{itemize}
%
\section{Scrittura del codice C\#}\label{sec:cap_sec_subsec}
Nel contesto dello sviluppo della parte back-end del progetto, è stato optato
l'impiego del linguaggio di programmazione C\#. Questa scelta si è rivelata
fondamentale per la gestione della logica e il comportamento del sito web.\
L'obbiettivo principale è stato garantire una gestione efficiente del database
direttamente attraverso il codice, sfruttando appieno le potenzialità del
framework Dapper. \\ Utilizzando C\# e il framework Dapper, è stato possibile
realizzare un corretto collegamento tra dati e utenti.\ Questo ha reso
possibile la generazione dinamica e accurata delle pagine web, consentendo
un'esperienza utente ottimale. \\ \\ Per eseguire delle query al database è
stato necessario creare una connessione dal codice C\# al database attraverso
la libreria \inlinecode{Microsoft.Data.SqlClient;}
%
%TODO: fixhere
\begin{algorithm}[H]
	\caption{classe per ottenere la stringa di connessione al database}
	\label{lst:genic_mpi}
	\begin{lstlisting}[label=lst:test]
	using Microsoft.Data.SqlClient;
	using Microsoft.EntityFrameworkCore;

	namespace OnBoarding.Services
	{
		public class ConnectionService
		private SqlConnection _connection = new SqlConnection();
		private SqlCommand _command = new SqlCommand();

		public static IConfiguration? Configuration { get; set; }

		public string GetconnectionString()
		{
			var builder = new ConfigurationBuilder().SetBasePath(Directory.GetCurrentDirectory()).AddJsonFile("appsettings.json");

			Configuration = builder.Build();
			return Configuration.GetConnectionString("ApplicationDBContextConnection");
		}


		public SqlConnection GetConnection() 
		{
			return _connection;
		}

		public SqlCommand GetCommand()
		{
			return _command;
		}

		}
	}
	\end{lstlisting}
\end{algorithm}
%
In questo modo, una volta che è stata stabilita una connessione, 
è stato reso possibile il processo di scrittura di query. A titolo esemplificativo, si può menzionare 
la query nella funzione denominata ``GetCourses'' presente all'interno del file ``CourseManager.cs''.\ Questa query, 
quando viene fornito l'ID dell'utente come input, ha lo scopo principale di recuperare e restituire 
l'insieme completo dei corsi associati all'utente specificato.
%
%TODO: fixhere
\begin{algorithm}[H]
	\caption{esempio funzione per l'esecuzione di una query da codice tramite il framework Dapper}
	\label{lst:genic_mpi}
	\begin{lstlisting}[label=lst:test]
	public List<CourseModel> GetCourses(string userid)
	{
		using (_connection = new SqlConnection(connectionService.GetconnectionString()))
		{
			_connection.Open();

			string sql = 
				@"SELECT DISTINCT Course.* 
				FROM Course, AspNetUsers, UserCourse 
				WHERE UserCourse.UserId = @userId     
				AND UserCourse.CourseId = Course.Id";
			var coursesList = _connection.Query<CourseModel>(sql, new { userId = userid }).ToList();

			_connection.Close();

			return coursesList;
		}
	} 
	\end{lstlisting}
\end{algorithm}
%
Si osserva che ciascun utente può essere associato a più corsi, 
stabilendo così una relazione uno a molti.\ Pertanto, è fondamentale 
che il valore restituito dalla funzione sia di tipo \inlinecode{List\textless CourseModel \textgreater}, 
in quanto questa struttura dati può contenere più di un corso associato a 
un utente specifico.\ Inoltre, grazie all'uso di Dapper, è possibile effettuare 
un'interrogazione al database mediante una stringa appositamente creata 
(come mostrato nelle righe 7\--11 del codice).\ Successivamente, è possibile 
ottenere il risultato dell'interrogazione corrispondente al parametro ``userid'' 
fornito alla funzione e al campo ``userId'' nella tabella ``Course'' (riga 12 del codice).
\\
Notare che siccome la funzione ritorna in questo caso una lista, il risultato della query
dovrà essere convertito in una collezione di oggetti dello stesso tipo (attraverso alla funzione di libreria \inlinecode{ToList()}).

\section{Layout della piattaforma}\label{sec:cap_sec_subsec}
\chapter{Risultati e sviluppi futuri}\label{chapter:formattazione}
%
%
\section{Risultati}\label{sec:cap_sec_subsec}
%
%* TODO: ADD INFO 
%
\section{Sviluppi fututi}\label{sec:cap_sec_subsec}
\subsection{Pagina statistiche}\label{sec:cap_sec_subsec}
La pagina delle statistiche deve subire un notevole potenziamento, mirando a offrire un'esperienza più completa ed esaustiva.\ 
Questo miglioramento deve concentrarsi sulla possibilità di confrontare una quantità più ampia di dati, 
garantendo al contempo un maggiore spazio dedicato all'analisi dei dati.\ 
Questa evoluzione si rivolge in particolare al personale del reparto HR dell'azienda, 
che ha la responsabilità di gestire il processo di OnBoarding per una varietà di utenti.
\\ \\
Grazie alla capacità di consentire un confronto più dettagliato dei dati, questa pagina deve diventare uno strumento prezioso per l'HR.\ 
Inoltre, dovrebbe essere possibile implementare un numero maggiore di grafici interattivi, che contribuiranno significativamente 
alla comprensione e all'analisi dei dati.
\\ \\
Questo potenziamento non solo agevolerà il processo decisionale dell'HR, ma permetterà anche di individuare con precisione le aree 
che richiedono interventi e miglioramenti.\ In definitiva, mira a trasformare la pagina delle statistiche in uno strumento 
fondamentale per l'ottimizzazione dell'OnBoarding all'interno del contesto dell'azienda.
%
\subsection{Ruoli utente}\label{sec:cap_sec_subsec}
Un'ulteriore evoluzione e arricchimento del sistema potrebbe senz'altro consistere nell'espandere 
l'attuale struttura dei ruoli, che attualmente comprende solo due categorie: ``User'' e ``Admin''.\ 
Questa espansione comporterebbe l'aggiunta di ulteriori ruoli, consentendo così una maggiore granularità nei 
permessi di visualizzazione all'interno delle pagine generate.
\\ \\  
L'implementazione di ruoli aggiuntivi potrebbe essere estremamente vantaggiosa, poiché permetterebbe di adattare 
l'accesso alle informazioni in base alle specifiche responsabilità e autorizzazioni di ciascun utente.\ 
In questo modo, si potrebbe garantire un maggiore controllo sull'accesso ai dati sensibili e una gestione 
più efficiente delle risorse aziendali.
\\ \\
Questo approccio fornirebbe ai decision-maker una maggiore flessibilità nella configurazione dei permessi e consentirebbe 
di assegnare ruoli intermedi, ad esempio ``Supervisor'' o ``Manager'', con diritti di accesso mirati alle informazioni rilevanti 
per il loro ruolo.\ Ciò migliorerebbe la sicurezza dei dati e l'efficienza operativa, 
consentendo a ciascun membro del team di accedere solo alle risorse necessarie per svolgere le proprie mansioni.
%
\subsection{Creazione dei Corsi}\label{sec:cap_sec_subsec}
Un aspetto che potrebbe essere considerato di importanza secondaria, ma che indubbiamente contribuirebbe al miglioramento 
significativo dell'esperienza utente, soprattutto dal punto di vista dell'amministratore, 
riguarda la possibilità di creare Corsi o CorsiTemplate in modo più efficiente, attraverso l'utilizzo di un nuovo processo/i di creazione.\ 
Attualmente, questa operazione richiede un processo manuale, ma esistono alternative 
che potrebbero semplificarne notevolmente l'implementazione.
\\ \\
Una di queste opzioni sarebbe consentire agli amministratori di importare direttamente i dati relativi ai Corsi o ai CorsiTemplate 
da fogli Excel.\ Questo approccio eliminerebbe gran parte del lavoro manuale, 
consentendo di caricare rapidamente una quantità significativa di informazioni nel sistema.
\\ \\
Inoltre, potrebbe essere utile considerare l'integrazione di funzionalità native all'interno del sito web per la creazione 
di Corsi o CorsiTemplate.\ Queste funzionalità integrate potrebbero offrire un ambiente più intuitivo e user-friendly 
per la progettazione e la gestione dei Corsi, migliorando ulteriormente l'esperienza dell'amministratore.
\\ \\
In definitiva, anche se questa caratteristica potrebbe essere considerata di importanza secondaria, 
l'implementazione di strumenti per l'importazione da Excel o funzionalità native all'interno del sito 
rappresenterebbe un passo avanti significativo nella semplificazione delle attività legate alla creazione di Corsi e CorsiTemplate, 
contribuendo in modo tangibile all'efficienza operativa complessiva e al miglioramento dell'esperienza dell'utente amministratore. 
\\ \\
\textit{Nota: } attualmente l'unica modalità per la creazione di nuovi Corsi e CorsiTemplate completi di relative Categorie e Step è la seguente:
\begin{enumerate}
    \item creazione del corso;
    \item apertura del corso;
    \item creazione della categoria;
    \item apertura della categoria;
    \item creazione dello step;
\end{enumerate}
%
\subsection{Text editor}\label{sec:cap_sec_subsec}
Attualmente, la feature che consente la scrittura in sintassi Markdown è piuttosto limitata e semplice.\
Potrebbe essere opportuno valutare l'implementazione di un vero e proprio editor di testo dedicato, 
in aggiunta alla possibilità di scrivere il Markdown manualmente.\ Questa aggiunta consentirebbe agli 
utenti di sfruttare gli stili di formattazione e i vantaggi offerti dalla sintassi Markdown in modo molto più facile, 
inclusivo ed intuitivo per tutti.\ Grazie alla presenza di appositi tasti funzione e a un ambiente di scrittura 
appositamente progettato, si potrebbero sfruttare al meglio tutte le potenzialità della formattazione Markdown 
senza la necessità di conoscere a fondo la sua sintassi.\ Questo migliorerebbe notevolmente l'esperienza degli 
utenti e renderebbe l'utilizzo di questa funzionalità ancora più accessibile e versatile.
\\
Di seguito un esempio preso dal programma ``Slack'':
\begin{figure}[ht]
	\centering
	\includegraphics[width=0.7\textwidth]{img/textEditor.png}
	\caption{esempio di un possibile text editor}
	\label{fig:one}
\end{figure}
%
%
\subsection{Reportistica errori e commenti}\label{sec:cap_sec_subsec}
\subsubsection{Reportistica errori}
Una possibile feature futura che potrebbe rivelarsi estremamente utile riguarderebbe 
l'integrazione di un meccanismo avanzato che consenta agli utenti incaricati di svolgere 
i corsi all'interno della piattaforma di segnalare errori o imprecisioni nei task assegnati.\ 
Questa innovativa funzionalità darebbe agli utenti la preziosa possibilità di comunicare direttamente 
agli amministratori eventuali problemi o lacune, come ad esempio link non funzionanti dovuti a modifiche 
nel corso del tempo o informazioni errate nelle assegnazioni.\ Inoltre, potrebbe includere anche la segnalazione 
di descrizioni mancanti o insufficientemente complete, offrendo un quadro completo delle aree da migliorare.
\\ \\
L'introduzione di questa avanzata funzionalità consentirebbe agli amministratori di ricevere feedback costanti 
da parte degli utenti, contribuendo così in modo significativo a elevare la qualità del servizio offerto.\ 
Inoltre, fornirebbe un meccanismo efficace per mantenere sempre aggiornati e corretti i contenuti presenti 
sulla piattaforma, garantendo agli utenti una esperienza di apprendimento completa e senza interruzioni.
\subsubsection{Commenti}
Potrebbe rivelarsi di di grande utilità considerare l'aggiunta di un sistema di assistenza diretta agli utenti 
tramite appositi commenti, in modo che gli utenti possano richiedere aiuto agli amministratori in maniera immediata.\ 
Sistema che potrebbe essere implementato in modo da consentire la comunicazione in threads dedicati, dove 
ogni nuovo messaggio può essere gestito in modo organizzato e pertinente.\ 
Questa funzionalità potrebbe essere integrata insieme alla feature di segnalazione degli errori o essere 
inserita in una sezione separata, per garantire un supporto completo e altamente efficiente per tutti gli utenti finali, 
migliorando notevolmente l'esperienza globale sulla piattaforma.
%
\subsection{Layout della piattaforma}\label{sec:cap_sec_subsec}
Attualmente, il layout della piattaforma si presenta con un design estremamente minimalista, 
concentrandosi esclusivamente su ciò che è strettamente necessario per la corretta visualizzazione 
dei dati e delle informazioni presenti nella piattaforma.\ Questo stile, sebbene efficiente, 
può essere considerato non particolarmente moderno.\ Pertanto, una delle possibili migliorie potrebbe 
consistere nell'apportare alcune modifiche significative al layout stesso, con l'obiettivo di migliorare l'esperienza dell'utente.
\\ \\
Una proposta consisterebbe nell'introduzione di navbar più complete e informative, specialmente per gli amministratori.\ 
Queste navbar potrebbero offrire un accesso rapido alle funzionalità e agli strumenti di amministrazione, 
semplificando così le attività di gestione e supervisione dell'intero sistema.
\\ \\
Inoltre, potrebbe essere benefico considerare l'implementazione di specifiche side-navbar per gli utenti, 
soprattutto quando si trovano nella pagina di visualizzazione dei corsi.\ 
Attualmente, la ricerca di task specifici da completare all'interno di un corso potrebbe risultare disorientante per l'utente finale, 
in quanto potrebbe richiedere uno sforzo eccessivo per individuare le informazioni rilevanti.\ 
L'aggiunta di una side-navbar dedicata potrebbe semplificare notevolmente questa operazione, 
consentendo agli utenti di accedere rapidamente ai alle categorie o ai task all'interno del corso senza difficoltà.
\\ \\
In conclusione, apportare modifiche al layout della piattaforma attraverso l'implementazione di navbar 
più esaustive per gli admin e di side-navbar specifiche per gli utenti, soprattutto nella pagina di visualizzazione dei corsi, 
rappresenterebbe un passo importante per migliorare l'usabilità complessiva del sistema e l'esperienza dell'utente finale.
%

%
%
%%%% Le Conclusioni
\pagestyle{plain}
\chapter*{Conclusione} %Se si cambia il Titolo cambiare anche la riga successiva così che appia corretto nell'conclusione
\addcontentsline{toc}{chapter}{Conclusione} %Per far apparire Introduzione nell'indice (Il nome deve rispecchiare quello del chapter)
%
In conclusione, durante il mio tirocinio, ho sviluppato un'applicazione web basata su .NET MVC 6.0 con 
l'obiettivo di creare un servizio che automatizzasse in modo efficiente il 
processo di integrazione dei nuovi dipendenti, un passaggio cruciale e ricorrente per 
tutti i nuovi membri dell'organico aziendale.\ Questo progetto ha consentito 
all'azienda di mettere a disposizione uno strumento estremamente automatizzato, riducendo 
significativamente il carico di lavoro del dipartimento HR e delegando gran parte delle 
attività operative alla gestione automatizzata del sistema.
%
%TODO: aggiungere grafico (a barre, con 2 barre: 1 il tempo medio precedente di gestione del processo, 2 il tempo medio di gestione del processo con il nuovo meccanismo), anche farlocco in cui mostro il miglioramento in termini di tempo nella gestione da parte degli amministratori rispetto alla modalità precedente 
%
\\ \\
Va notato che l'azienda, in virtù della sua certificazione ISO 9001, non solo è obbligata a farlo, 
ma si impegna attivamente a gestire e analizzare attentamente i feedback provenienti dagli utenti del sistema, 
con l'obiettivo di migliorarne costantemente l'efficacia e l'efficienza.
\\ \\
L'applicazione web, pur rispettando appieno tutti i requisiti aziendali iniziali, 
offre un ampio margine per ulteriori miglioramenti, come discusso dettagliatamente nel capitolo dedicato del documento~\ref{chapter:Risultati_e_sviluppi_futuri}, 
offrendo così opportunità significative per un futuro sviluppo e ottimizzazione.
%
%%%% La bibliografia
% \bibliographystyle{apalike} %{plain} -- Scegliere lo stile preferito
% \cleardoublepage
% \addcontentsline{toc}{chapter}{\bibname}
%\bibliography{./Bibliografia}
%
\chapter*{Ringraziamenti}
Desidero ringraziare sinceramente le seguenti persone dell'azienda ISolutions che mi hanno 
accompagnato e guidato durante l'esecuzione del mio progetto di tirocinio:
\begin{itemize}
    \item Alessandro Bardini (Line Manager) \& Gialuca Bellini (Line Manager \-- NOC Team Lead).\ Per la loro preziosa guida nell'ambito dell'esecuzione dei vari compiti del progetto.
    \item Ivan Anselmi (Architect, R\&D Dev).\ Per l'assistenza nell'applicazione del servizio di Autenticazione e nella corretta gestione delle query string per la realizzazione della piattaforma web sviluppata.
    \item Roberto Isca (Senior Web Designer \& Web Developer).\ Per i preziosi consigli e l'aiuto fornito in numerosi problemi di layout.
    \item Marco Chiesa.\ Per aver reso possibile il mio lavoro in remoto grazie alla configurazione della VPN e dei servizi aziendali.
\end{itemize}
%
Ringrazio, inoltre, coloro che mi hanno fornito preziosi consigli e che mi hanno sostenuto sia durante il progetto di tirocinio sia nella stesura di questo elaborato:
\begin{itemize}
    \item Donatello Larocca.
    \item Mariachiara Michelini.
\end{itemize}
%
Il loro sostegno e la loro guida sono stati fondamentali per il successo del mio tirocinio. Grazie di cuore a tutti loro.
%
% Le appendici
\appendix
\include{Capitoli/Appendici/Appendice1}
%
\end{document}